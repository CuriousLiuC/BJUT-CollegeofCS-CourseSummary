\documentclass[UTF8]{ctexart}  %使用中文版的article文档类型排版,并选择UTF8编码格式
\usepackage{amsmath}  %使用宏包,这里使用的是调用公式宏包,可以调用多个宏包
\usepackage{geometry}
\geometry{left=31.7mm,right=31.7mm,top=25.4mm,bottom=25.4mm}
\begin{document}  %开始写文章


\title{BJUT操作系统学期总结}  %大括号里填写标题
\author{Curious}  %大括号里填写作者姓名
\date{\today}    %大括号里填写\today会自动生成当前的日期
\maketitle     %我们写了以上内容以后一定要添加这个,制作标题,否则上面的内容都是无效的。

\large{操作系统可以说从课程广度深度,还有期末考试难度,平均分来看都是大三上学年最难的一科,主要原因还是因为操作系统课程知识点太多,而且课上时间有限,实验进一步压缩了课内时间,可以说如果大三上想要获得比较理想的分数,操作系统课是值得重视的科目之一,再加上作为几乎所有学校计算机专业考研的科目,在复习的时候也可以更加轻松。

另外一个导致这门课程学习起来比较困难的原因是,对于几个计算机类的专业来说,大三上是大学来事情最多的一个学期,可能要是专注于操作系统的学习不难,但是如果很多事情混在一块,总会分散精力,希望这篇可以帮助到北工大学习操作系统的人。

操作系统除了课表上安排的课程以外,还包含4次上机实验,还有不确定的上机考试,总体来说除了试验外对课外时间占用不多,但是要想学好这一科,课外付出的时间肯定会是所有期末考试周科目中最多的。
}
\tableofcontents  %表示目录部分开始

\newpage
\section{操作系统理论学习}%目录的前缀页面都会自动排版不需要手动排版
\large{在课上老师可能有一些随堂练习选择题,但可能不同老师上课有不同的操作。每次至少做到课后进行复习,如果时间比较紧可以比如下次上课前一天对上次课学的进行一个复习,复习效果就看自己每次选择题答的怎样。maple上也会有一些选择题,可以用来检验自己平时的学习。这些题如果能在平时学的时候就弄8-9成懂,期末基础性的题基本就没啥问题了,期末复习过程中来看,除了小部分题之外,其他都是简单-中档题。

课上ppt一定要及时下载,对着复习,操作系统来说每页ppt上可能只有几个概念性的词,要想得更高分就得理解这些词后边的概念。}

\newpage
\section{操作系统实验}
\large{实验因为有中国大学mooc上的慕课视频还是很简单的,基础实验来说跟着慕课一点一点学,基本就够用了,注意要了解慕课中用到的函数具体参数是什么,怎么用,老师在最后检察环节时可能会挑一些比较常见的函数,例如execlp,execvp来问里头每个你用到的参数代表的意义。另外如果有时间可以稍微熟悉一下linux manpage的使用,说实话我也不太会,但是最后在检查的时候和上机考试的时候可能会用到。

检查要尽量早检查,到后边人多了检查不太方便,因为老师问的会非常细,也可能会影响最后的分数,早点做完之后去单独联系老师检查也可以。

上机考试的话我们这年是自愿的,估计以后可能会变成强制的了,就算是自愿也强烈建议参加,因为除了题目本身的分以外还会有一些代码风格什么的分,不会因为题目不对就一分不算的。我做的是一个PV操作的题,用代码实现,因为实验里有类似的题所以还挺方便的。另外一波同学我有些记不清了,是一个比较复杂的题了,好像和实验指导书中最后那个接受命令的东西有点关系,sigint那些的,貌似还要利用到manpage,遇到这种题也不用太慌,大家都不太会尽力就行了。}

\newpage
\section{操作系统期末}
\large{期末首先的建议是把网上能找到的几套卷子(我也放在这个目录下了)都刷的必须100\%明白,尤其是简答题和大题。简答题几套刷下来以后,大概就能明白一些常问的问题。

简答题例如spooling,我们这年考试的时候spooling问的题目是小明有线单独连接一台打印机是否是spooling,还有大家共享一台打印机,spooling的过程是什么。答案我也不是很清楚,大家可以问老师或者自己讨论。

简答题在我们这一年没有考和通过tlb命中率等计算时间的有关的几个公式,在选择题考了这方面的,导致时间非常紧张。磁盘调度算法,调度了多少个磁道,也是在小题考的。

简答题还有让我印象比较深刻的题就是问一个用户操作的序列过程里,进程经历了哪些状态的转换,就是五状态转换那张图,这个我记得几套卷子里有类似的题,也要弄明白这个类型的问题。

大题一个是先自己归类出有哪些范围,有个复习ppt可以找老师要一下,不是自己的也不是网上公开发的东西我就不放在博客里了,maple或者找学长学姐要一下,期末时候问老师老师也会给。和I/O操作有关的题目要弄的非常明白。

说几个让我印象比较深的题:
\begin{itemize}
  \item [1)] 
  缺页中断调页那个画每次调页情况和缺页次数的,放在这里的几套卷子都只问了缺页中断次数,我们当时考试的时候问了和这个题有关的I/O次数,可以问问和老师有关的。     
  \item [2)]
  有一个CPU调度有关的题,除了几种常见的CPU调度算法一定要掌握外,考试的时候给了一个比较特殊的调度方式,每次运行一个时间片后,优先级会-1之后在进行一次调度。会让写一个表格,当时考试给了一个到现在也不太明白的说法:x时刻后,这个时刻后如果考试不明白一定要及时问问,或者考前几天找老师确认一下。这个题就是别慌,慢慢弄。
  \item [3)]
  文件I/O有关的题会有一道,和文件一章的作业比较像,就是最少几次,最多几次那个图,但是没有问最少几次I/O最多几次I/O,问的是类似有关的题,把模板题搞好即可。
  \item [4)]
  银行家算法那个题里,一定要搞清allocate,request,need等矩阵的互相之间关系,注意题目给出的条件和几套卷子里刷到的一样不一样,我们考试时候题目没有给free矩阵,而是给了总资源数的矩阵,需要通过总-其他什么什么的得到free矩阵。找安全序列什么的倒还都是老套路。
  \item [5)]
  PV操作题是必考,我们考的非常贴近实际应用,是一个生产者消费者问题的变型,问的是ACM竞赛生产多个颜色气球,拿气球的,还有对一血黑气球的处理,原题已经记不清了,可以自己用这个背景出一个然后试试。
\end{itemize}
}

\newpage
\section{备注部分}
\large{操作系统我们这年所有考试结束后才出分,等待的时候不用太着急,另外有大佬不是做了一个定制发送分数的服务,那个实在厉害。如果只以分数论的话,操作系统还是很有难度,我的分也不是很高,仅供大家参考。

书写文档的过程也是我对LaTex学习的过程,希望看到此篇的人能够加上自己对操作系统这一门课程的看法,继续完善此总结。}
\end{document}  %结束写文章